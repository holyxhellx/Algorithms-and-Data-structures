\documentclass{tufte-handout}
\usepackage{tikz}
\usepackage{booktabs}
\usepackage{siunitx}
\input{vc.tex}  % For version control

\title{Giantbook}
\author{}
\date{\GITAuthorDate, rev. \GITAbrHash}

\begin{document}
\maketitle
%\section{}
%\subsection{}

\begin{abstract}Study the structure of social networks in the
  Erd\H{o}s-Renyi model, such as the emergence of the giant component. 
  This exercise serves as a very simple client for
  the Union--Find data structure.
\end{abstract}

\section{Description}
A \emph{social network} consists of \emph{individuals}, some of which are \emph{linked}. 
You are welcome to think of the structure of ``friends'' defined by FaceBook.
In particular, the social networks studied in this exercise are
symmetric in the sense that if $p$ is connected to $q$ then $q$ is connected
to $p$.
(Other networks, such as the structure of ``followers'' on Twitter, are
not symmetric.)

\begin{marginfigure}
\tikzset{vertex/.style={minimum size = 3pt, inner sep = 0pt, circle,fill = black}}

\begin{tikzpicture}
\node at (-1,0) {$T=0$};
\foreach \a in {0,60,...,300} \node [vertex] (\a) at (\a:.3) {};
\end{tikzpicture}

\bigskip
\begin{tikzpicture}
\node at (-1,0) {$T=1$};
\foreach \a in {0,60,...,300} \node [vertex] (\a) at (\a:.3) {};
\draw(60)--(300);
\end{tikzpicture}

\bigskip
\begin{tikzpicture}
\node at (-1,0) {$T=2$};
\foreach \a in {0,60,...,300} \node [vertex] (\a) at (\a:.3) {};
\draw(60)--(300);
\draw(0)--(120);
\end{tikzpicture}

\bigskip
\begin{tikzpicture}
\node at (-1,0) {$T=3$};
\foreach \a in {0,60,...,300} \node [vertex] (\a) at (\a:.3) {};
\draw(60)--(300);
\draw(0)--(120);
\draw(180)--(240);
\end{tikzpicture}

\bigskip
\begin{tikzpicture}
\node at (-1,0) {$T=4$};
\foreach \a in {0,60,...,300} \node [vertex] (\a) at (\a:.3) {};
\draw(60)--(300);
\draw(0)--(120);
\draw(180)--(240);
\draw(60)--(0);
\end{tikzpicture}

\bigskip
\begin{tikzpicture}
\node at (-1,0) {$T=5$};
\foreach \a in {0,60,...,300} \node [vertex] (\a) at (\a:.3) {};
\draw(60)--(300);
\draw(0)--(120);
\draw(180)--(240);
\draw(60)--(0);
\draw(300)--(0);
\end{tikzpicture}

\bigskip
\begin{tikzpicture}
\node at (-1,0) {$T=6$};
\foreach \a in {0,60,...,300} \node [vertex] (\a) at (\a:.3) {};
\draw(60)--(300);
\draw(0)--(120);
\draw(180)--(240);
\draw(60)--(0);
\draw(300)--(0);
\draw(240)--(0);
\end{tikzpicture}

\bigskip
\caption{A tiny evolving network of 6 individuals.
 The \emph{nonisolated} event happens at $T=3$, the giant component emerges at $T=4$, and the whole network becomes connected at $T=6$.} 
\end{marginfigure}

We need some standard terminology from network theory:
An individual is \emph{isolated} if it not link to any other individual.
Two individuals $u$ and $v$ are \emph{connected} if there is a sequence of indviduals $v_0,\ldots,v_k$ with $v_0=u$ and $v_k=v$ such that $v_{i-1}$ and $v_i$ are linked for each $i\in\{1,\ldots,k\}$.
A \emph{connected component} is a maximal subset of connected individuals.

We are interested in the behaviour of networks as they grow when links are added at random.
In particular, we want to know at which times the following events occur:
\begin{description}
\item[nonisolated]
  A network becomes \emph{nonisolated} when it has no isolated individuals.
\item[giant component] A network has a \emph{giant component} when
  it contains a connected component with at least half the individuals.
\item[connected] The network becomes connected when every individual is connected
  to all other individuals.
\end{description}
  We study the dynamics of such a network in a very simple model
  called the \emph{Erd\H{o}s-Renyi model}.
  In this model, there are $N$ individuals, all are isolated in the beginning, and links are formed completely at random.
  We want to know when the three events occur, and in which way the timing
  depends on $N$, the initial number of individuals.

  Let me be very specific about the random model:
  In each round, select two individuals $u$ and $v$ uniformly at random.
  If $u\neq v$ and $u$ and $v$ are not yet linked, then create a link between them.

\subsection{Deliverables}
\begin{enumerate}
\item Your implementation of {\tt GiantBook.java}.
\item A modified Union--Find data structure in a separate file.
	Call this {\tt MyUnionFind.java}. 
\item A project report in PDF, you are strongly encouraged to use the skeleton on the last page.
\end{enumerate}

It is important that the functionality of the union--find data
structure and of the client ({\tt GiantBook.java}) are clearly
separated.
In particular, don't put everything in one file.
Your extensions of the functionality of the abstract data type for union--find should be formulated in terms of meaningful, general methods that do not refer to the functionality of the client. 
\footnote{For instance, {\tt boolean hasGiantComponent()} is not a good method for your union--find data type, because it needs to know how a giant component is defined in this particular application. On the other hand, {\tt int maxComponentSize()} would be a perfectly fine method for a union--find data type that could be useful to many other clients as well.}

\section{Tips}

The simplest event to simulate is \emph{connected}, because
the Union--Find structure in the book already tells you if there is
just one component.
Therefore, I suggest you completely solve the whole exercise for this
event first, including filling out the relevant table column in the report.
The book's standard library has useful methods on page 30.
Make sure it works correctly for very small instances ($N=10$) by
following the computation step by step (print out what happens, check
it by hand).

After that, study the emergence of the giant component.
This involves modifying the Union--Find implementations in the book.
I needed four lines of code.

Finally, study the nonisolated individuals.
There is an obvious way of changing the Union--Find structure to detect
the existence of isolated individuals (simply run through all elements
and see if the size of their component is $1$), but that will probably
be too slow to be useful for large $N$.
You need to do something that is both simpler and much faster.
Again, I can do this in four lines of code.

Oh, and one more thing: it's probably a good idea to print the values
using scientific notation, \emph{i.e.}, like {\tt 6.96e+02}.
You can use the {\tt StdOut.printf} method with format string
``{\tt \%.2e}''  for that.

\subsection{Questions}
\begin{description}
\item[My programme is slow.] You should be able to detect the emergence
  of a giant component for $N=10,000,000$ within seconds on a modern
  machine.
  If not, something is wrong.
  I am using weighted quick-union.
\item[What about those standard deviations?]
	Reporting the result of random experiments using means and standard deviations is standard scientific practice.
  When performing experiments that involve randomness,
  the results will depend on the random choices.
  Therefore we can't just run your experiment once and report the result.
  Instead, the experimented is repeated $T$ times, and we report the average
  (or ``mean'') of those experiments.
  So how large should $T$ be?
  That's where the standard deviation comes in.\sidenote{Read Wikipedia's entry on standard deviation for more.}
  This is not a course in statistics, so we won't make a big deal out
  of this, but your standard deviation should be an order of magnitude
  smaller than the mean value you report.
  For example, if you report mean 536, your standard deviation should
  be at most something like 63, but not 251.
  Otherwise you need to increase $T$.
  For small experiments (say, $N=10$) this is not important.
  The standard library has methods for both computing both the mean and standard
  deviation of a sequence of values {\tt StdStats}, so you don't write any code for this.
\end{description}

\section{Perspective}

Want to learn more?
You can read about the Erd\H{o}s--R\'enyi model or more advanced models for social networks like the Watts--Strogatz model at Wikipedia.
You can change your simulation to use that model instead, for example, and see what happens.

If you don't like the random graph models, check out the Stanford Network Analysis Project for real life data with large datasets.
For example, running your code on the (famous) Enron email network should give the same value for the size of the largest component (called WCC in the dataset statistics table.)

\newpage
\section{Giantbook Report}


by Alice Cooper and Bob Marley\sidenote{Complete the report by filling
  in your names and the parts marked $[\ldots]$.
  Remove the sidenotes in your final hand-in.}


\subsection{Results}

The following table summarises our results.
It shows the average number of random connections needed before the emergence of the giant component (``giant''), the disappearance of the last isolated individual (``no isolated''), and when the network becomes connected (``connect'').

\medskip
\begin{fullwidth}
\begin{tabular}{rcccccccc}\toprule
$N$ & $T$ & giant & (stddev) & no isolated & (stddev) & connected & (stddev)\\\midrule

     100 & 100 & \num{7.15e+01} &5.77 &    \num{2.68e+02} & \num{6.32e+01}  &\num{2.69e+02} & \num{6.3e+01} \\
    1000 \\
   \num{e4} \\
  \num{e5} \\
 \num{e6} \\
\num{e7} \\\bottomrule
\end{tabular}
\end{fullwidth}

\medskip\noindent
Our main findings are the following:
The first thing that happens is that $[\ldots]$ \sidenote{Replace by ``the network becomes connected'' or ``the giant component emerges'' or ``the last invididual becomes nonisolated'', whatever is correct.}, which happens at a time $[\ldots]$ \sidenote{Replace by ``linear'' or ``logarithmic'' or ``quadratic'' or something like that.} in $N$.
Perhaps surprisingly, two of the events seem to happen simultaneously, namely $[\ldots]$ and $[\ldots]$, which happen at time $[\ldots]$ in $N$.


\subsection{Implementation details}

We have based our union--find data type on $[\cdots]${\tt .java} from Sedgewick and Wayne: {\em Algorithms, 4th ed.}, Addison--Wesley (2011).
We added a method $[\ldots]$ by adding the following lines to $[\dots]$:\footnote{And so on. Describe your modified union--find data type.
You are encouraged to include code in this report; hopefully you added just a handful of lines and can explain everything needed to understand your changes in a single paragraph, maybe two. 
You don't need to explain your \emph{client} code at all.}

Assuming we never run out of memory or heap space, if we would let our algorithm for detecting the emergence of a giant component run for 24 hours, it could compute the answer for $N= [\ldots]$.

We've run the code using a quick-find implementation as well.
In 1 hour, we were able to handle and instance of size $N=[\ldots]$.



\subsection{Discussion}

We defined the giant component to have size at least $\alpha N$ for $\alpha = \frac{1}{2}$. \footnote{Remove and/or complete one of the following two sentences.}
The choice of $\alpha$ is arbitrary, choosing other constants (such as $\alpha=\frac{1}{10}$ or $\alpha=\frac{9}{10}$) gave essentially the same results.
The choice of constant is important; choosing $[\ldots]$ changes the experiment completely because $[\ldots]$.

\end{document}  
