\documentclass{tufte-handout}
\usepackage{amsmath,amsthm}

\input{vc.tex}  % For version control

\usepackage{booktabs}
\usepackage{graphicx}
\usepackage{tikz}
\usepackage{listings}

\newtheorem{claim}{Claim}[section]
\title{Connected Components Warmup}
\date{\GITAuthorDate, rev. \GITAbrHash}
\author{}

\begin{document}
\maketitle

\subsection{Description}

This is the first assignment in BADS/SGDS. 
It is supposed to be very simple; the main objective is to form groups, set up your programming environment, download the textbook's code, find a way to include the book's libraries in you java environment.
The algorithmic problem itself is very simple and basically already solved in section 1.5 of the textbook:
Form pairwise connections between sites and at the end report if sites 0 and 1 are in the same component.

\subsection{Input}

The input contains an integer (the number of sites) followed by a sequence of pairs of integers (connections), like the book's example files {\tt tinyUF.txt}, {\tt mediumUF.txt}, and {\tt largeUF.txt}.

\marginpar{
The contents of {\tt tinyUF.txt}:\\[1ex]
\begin{tt}
10\\
4 3\\
3 8\\
6 5\\
9 4\\
2 1\\
8 9\\
5 0\\
7 2\\
6 1\\
1 0\\
6 7
\end{tt}
}

\subsection{Output}

The string {\tt true} if 0 and 1 belong to the same connected component after forming all connections in the input. 
Otherwise {\tt false}.

\subsection{Example}

\begin{quotation}
\begin{verbatim}
% java UFW < tinyUF.txt
true
\end{verbatim}
\end{quotation}

\subsection{Requirements}

Your class has to be called UFW and reside in a file called {\tt UFW.java}.
Your class must inlude a single, static main method and nothing else.
It should refer to one of the union--find data structures from the books; download that code and put it into the same directory as {\tt UFW.java}.
Do not modify that file.
Your solution must use as much code from the book as possible---the exercise can be solved by copy--paste and writing one new line of code.
You have to use {\tt StdOut.println} for output and {\tt StdIn.readInt()} for input.
Your program must communicate with standard input and output from the command line exactly as shown.

\subsection{Deliverables}

Upload exactly two files:

\begin{enumerate}
\item the source code of {\tt UFW.java}
\item A report in PDF. 
  Use the report skeleton on the next page.
\end{enumerate}

\subsection{Tips}
The main difficulty in this exercise is to convince your java setup to find the {\tt StdOut} and {\tt StdIn} classes in the book's standard library.
This is the main point of this exercise, so take it seriously. 

In fact, before starting with the exercise itself, try to make the following program run:

\begin{quotation}
\begin{verbatim}
public class Hello
{
  public static void main (String[] args)
  {
    StdOut.println("Hello world!");
  }
}
\end{verbatim}
\end{quotation}

Once that works, change {\tt Hello.java} so that reads your name (in a text file) from Standard Input and returns it to you.
The functionality should be like this:

\begin{quotation}
\begin{verbatim}
% cat myname.txt
Bob
% java Hello < myname.txt
Hello Bob!
\end{verbatim}
\end{quotation}

Do \emph{not} avoid the issue by simply copying {\tt StdOut.java} into you current directory, or using {\tt System.out} and {\tt System.in}.
There are several suggestions for how to set up the book's standard library on the book website, under \emph{Web Resources, Code}.
You are \emph{strongly encouraged} to use the setups called Windows Command Prompt (manual) or Mac OS X Terminal (manual) or Linux Command Line (manual), depending on which architecture you are on.

\newpage
\section{Report: Connected Components Warmup}


by Alice Cooper and Bob Marley\sidenote{Complete the report by filling
  in your names and the parts marked $[\ldots]$.
  Remove the sidenotes in your final hand-in.}


\subsection{Results}

\begin{tabular}{ll}
  \toprule
  input file & 0 connected to 1? \\
  \midrule
  {\tt tinyUF.txt} & yes \\
  {\tt mediumUF.txt} & $[\ldots]$ \\
  {\tt largeUF.txt} & $[\ldots]$ \\
  \bottomrule
\end{tabular}

\subsection{Methods}

For the union--find data structure we used  $[\ldots]$\footnote{Replace by {\tt QuickFindUF.java} or whatever you actually used.}
from the textbook. 
$[\dots]$.\footnote{Add a single sentence motivating your choice of data structure. 
For instance ``Using blabla instead turned out to take more that 5  blabla on instance blabla.''
Or ``It didn't make much of a difference.''
Or ``We didn't have time to try anything else''.}


\end{document}
 